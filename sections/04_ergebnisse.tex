\section{Ergebnisse: Die Fallstudie \enquote{Trink-Genosse eG}}

\subsection{Allgemeine Vorstellung}

Die Trink-Genosse eG wurde als eingetragene Genossenschaft gegründet und betreibt die Bar \enquote{TRINK-Genossin} im Kölner Stadtteil Ehrenfeld. Sie bezeichnet sich selbst als \enquote{Kölns erste genossenschaftliche Bar} und versteht sich als Ort für vielfältige soziale und kulturelle Aktivitäten mit dem expliziten Ziel der Demokratieförderung.

Die Genossenschaft organisiert sich über demokratische Strukturen mit einer jährlichen Generalversammlung zur Wahl des Aufsichtsrats und Entlastung des Vorstands sowie einem monatlichen Plenum, in dem basisdemokratisch über operative und strategische Fragen entschieden wird. Digitale Kommunikationskanäle wie RocketChat ermöglichen eine kontinuierliche Diskussion und Meinungsbildung zwischen den Mitgliedern.

\subsection{Analyse des Geschäftsmodells nach dem Morphologischen Kasten}

Basierend auf dem theoretischen Rahmen von \textcite{blome-dreesGenossenschaftlicheGeschaeftsmodelleSemantik2023} lassen sich die Merkmale des Geschäftsmodells der Trink-Genosse eG analysieren.

\begin{sidewaystable}[htbp]
\centering
\small
\setlength{\tabcolsep}{4pt}
\renewcommand{\arraystretch}{1.3}
\caption{Morphologischer Kasten genossenschaftlicher Geschäftsmodelle. Eigene Darstellung.}
\label{tab:morphologischer_kasten}
\begin{tabularx}{\textwidth}{|p{0.18\textwidth}|p{0.20\textwidth}|p{0.20\textwidth}|X|p{0.08\textwidth}|}
\hline
\textbf{Merkmalsgruppe} & \textbf{Ordnungsmerkmal} & \textbf{Einzelmerkmal} & \textbf{Ausprägungen} & \textbf{Mischformen} \\
\hline
\multirow{14}{*}{Sinnbezogene Merkmale} & \multirow{8}{*}{Mitglieder/Nutzer (\enquote{Wer?})} & Leistungsadressaten & Mitglieder; Dritte; Gesellschaft & Ja \\
\cline{3-5}
 &  & Identit\"atsprinzip & Ja; Nein\\[-2pt]
 &  &  & Eigent\"umer \& Nutzer; Eigent\"umer \& Besch\"aftigte\\[-2pt]
 &  &  & F\"ordergenossenschaft; Produktivgenossenschaft & Nein \\
\cline{3-5}
 &  & Gesch\"aftsbeziehung & Hauptzweck; Nebenzweck\\[-2pt]
 &  &  & Mitgliedergesch\"aft; Nichtmitgliedergesch\"aft & Ja \\
\cline{3-5}
 &  & Tr\"agerschaft & Privat; Staatlich & Ja \\
\cline{3-5}
 &  & Betriebsformen & Haushalte; Unternehmen &  \\
\cline{2-5}
 & \multirow{6}{*}{Nutzenversprechen (\enquote{Was?})} & R\"aumliche Verankerung & Lokal; Regional; \"Uberregional; National; International & Ja \\
\cline{3-5}
 &  & Leistungsarten & Wirtschaftlich; Sozial\\[-2pt]
 &  &  & G\"uter; Dienstleistungen\\[-2pt]
 &  &  & Produktion; Bezug; Absatz & Ja \\
\cline{3-5}
 &  & Schl\"usselaktivit\"aten & \"Okonomisierung; Vertretung; Koordinierung & Ja \\
\cline{3-5}
 &  & Funktions\"ubernahme & Eine Funktion; Mehrere Funktionen & Ja \\
\hline
\multirow{10}{*}{Strukturbezogene Merkmale} & \multirow{6}{*}{Wertsch\"opfungsarchitektur} & Kooperationspartner & Verbundinterne Kooperationspartner; Verbundexterne Kooperationspartner\\[-2pt]
 &  &  & Finanzielle Beteiligung; Nicht-finanzielle Beteiligung & Ja \\
\cline{3-5}
 &  & Vertriebskan\"ale & Eigene Vertriebskan\"ale; Fremde Vertriebskan\"ale\\[-2pt]
 &  &  & Analog; Digital\\[-2pt]
 &  &  & Einkanalstrategie; Multikanalstrategie; Omnikanalstrategie\\[-2pt]
 &  &  & Filialen; Vertriebsabteilungen; Online-Shop; Plattformen & Ja \\
\cline{2-5}
 & \multirow{4}{*}{Ertragsmechanik (\enquote{Wert})} & Ressourcen & Materiell; Immateriell\\[-2pt]
 &  &  & Sachkapital; Finanzkapital; Sozialkapital; Humankapital & Ja \\
\cline{3-5}
 &  & Erl\"osmodell & Umsatzerl\"ose; Beteiligungserl\"ose; Regelm\"a\ss{}ige Beitr\"age; Subventionen & Ja \\
\cline{3-5}
 &  & Kostenmodell & Fixkosten; Variable Kosten & Ja \\
\hline
\end{tabularx}
\end{sidewaystable}


\subsubsection{Mitglieder/Nutzer (\enquote{Wer?})}

\paragraph{Leistungsadressaten}
Die Genossenschaft richtet sich primär an ihre \textbf{Mitglieder}, öffnet ihr Angebot aber bewusst für \textbf{Dritte} (\enquote{Nachbarn, Freunde und Neugierige}). Das monatliche Plenum und die dort beschlossenen Aktionen, wie das \enquote{Soli-Café}, das Initiativen eine Plattform bietet, unterstreichen den Willen, über die reine Mitgliederförderung hinaus einen gesellschaftlichen Mehrwert zu schaffen.

\paragraph{Identitätsprinzip}
Das Identitätsprinzip wird stark gelebt, primär als \textbf{Eigentümer \& Nutzer}. Die Mitglieder sind Kapitalgeber und zugleich Gäste der Bar. Darüber hinaus existiert eine ausgeprägte Form von \textbf{Eigentümer \& Beschäftigte}. Dies umfasst sowohl Mitglieder, die als Aushilfen angestellt sind, als auch einen Kern sehr engagierter Mitglieder, die ehrenamtlich in Arbeitsgruppen, im Plenum oder durch Bar-Schichten die Genossenschaft tragen. Es gibt jedoch auch Angestellte, die keine Mitglieder sind, was die strikte Identität leicht aufweicht.

Die Genossenschaft weist somit Charakteristika sowohl einer \textbf{Fördergenossenschaft} als auch einer \textbf{Produktivgenossenschaft} auf.

\paragraph{Geschäftsbeziehung}
Der \textbf{Hauptzweck} ist das Mitgliedergeschäft, das durch eine gelebte demokratische Struktur geprägt ist. Das \textbf{Nichtmitgliedergeschäft} (regulärer Barbetrieb) dient als \textbf{Nebenzweck}, um die laufenden Kosten zu decken und die finanzielle Stabilität für die Mitgliederförderung zu sichern.

\paragraph{Trägerschaft und Betriebsformen}
Die Trägerschaft ist \textbf{privat} durch private \textbf{Haushalte}.

\paragraph{Räumliche Verankerung}
Die Genossenschaft ist bewusst \textbf{lokal} im Kölner Stadtteil Ehrenfeld verankert. Sie versteht sich als Ort für die Nachbarschaft und ist ein klares Beispiel für eine Bürgergenossenschaft, die eine soziale und kulturelle Infrastruktur für ihr direktes Umfeld schafft.

\subsubsection{Nutzenversprechen (\enquote{Was?})}

\paragraph{Leistungsarten}
Das Nutzenversprechen ist eine Mischung aus \textbf{wirtschaftlichen} und \textbf{sozio-kulturellen} Leistungen. Wirtschaftlich bietet sie \textbf{Dienstleistungen} in Form von Gastronomie, Catering und Eventvermietung. Der sozio-kulturelle Nutzen ist jedoch mindestens ebenso wichtig: Sie ist ein Ort der Begegnung, der Demokratieförderung und des sozialen Engagements. Ein eigener Fördertopf zur Unterstützung anderer Genossenschaftsgründungen zeigt, dass die Förderung der genossenschaftlichen Idee selbst Teil des Nutzenversprechens ist.

Funktional steht die \textbf{Dienstleistungen} im Vordergrund.

\paragraph{Schlüsselaktivitäten}
Die Aktivitäten umfassen alle drei Dimensionen:
\begin{itemize}
\item \textbf{Ökonomisierung:} Der Betrieb der Bar zur Deckung der Kosten
\item \textbf{Koordinierung:} Die Organisation der demokratischen Prozesse (Plenum, Generalversammlung), der Arbeitsgruppen und der ehrenamtlichen Arbeit
\item \textbf{Vertretung:} Das aktive Eintreten für die genossenschaftliche Idee und die Schaffung eines Modells für alternative, demokratische Wirtschaftsformen
\end{itemize}

\paragraph{Funktionsübernahme}
Als \textbf{Mehrzweckgenossenschaft} vereint sie die Funktionen eines Gastronomiebetriebs, eines Kultur- und Veranstaltungsortes und einer Plattform für zivilgesellschaftliches Engagement.

\subsubsection{Wertschöpfungsarchitektur (\enquote{Wie?})}

\paragraph{Kooperationspartner}
Die Genossenschaft ist gut vernetzt und arbeitet mit \textbf{verbundexternen Partnern}. Die Kooperationen sind oft ideeller Natur, wie die Unterstützung des Wohnprojekts \enquote{Petershof} durch den Kauf von Genossenschaftsanteilen oder das Sponsoring des Sportvereins \enquote{Roter Stern}. Dies zeigt sowohl \textbf{finanzielle} als auch \textbf{nicht-finanzielle Beteiligung}, dass Partnerschaften nicht nur aus ökonomischem Kalkül, sondern auch aus gemeinsamer Wertorientierung eingegangen werden.

\paragraph{Vertriebskanäle}
Eine \textbf{Multikanalstrategie} kombiniert \textbf{analoge} (die physische Bar) und \textbf{digitale} Kanäle. Die Website dient als Informationsplattform, während interne Tools wie RocketChat entscheidend für die Organisation, Kommunikation und damit für die Wertschöpfung sind. Alle Kanäle sind \textbf{eigene Vertriebskanäle}.

\subsubsection{Ertragsmechanik (\enquote{Wert?})}

\paragraph{Ressourcen}
Die wichtigste Ressource ist das \textbf{immaterielle Kapital}. Das \textbf{Sozialkapital} (Gemeinschaft, Vertrauen, Netzwerk) und das \textbf{Humankapital} (Wissen, Engagement und ehrenamtliche Arbeit der Mitglieder) sind die eigentlichen Treiber der Genossenschaft. Das \textbf{Finanzkapital} speist sich hauptsächlich aus den Genossenschaftsanteilen der Mitglieder sowie aus Fördermitgliedschaften. Das \textbf{Sachkapital} umfasst die Bar-Ausstattung.

\paragraph{Erlösmodell}
Das Modell ist diversifiziert und umfasst:
\begin{itemize}
\item \textbf{Umsatzerlöse} aus dem Barbetrieb (Haupteinnahmequelle)
\item \textbf{Spenden/Einmalzahlungen} (Genossenschaftsanteile, Solibeiträge)
\end{itemize}

Eine Ausschüttung von Dividenden ist derzeit nicht vorgesehen, was den Fokus auf die ideelle Förderung unterstreicht.

\paragraph{Kostenmodell}
Das Modell besteht aus \textbf{Fixkosten} (Miete, Gehälter) und \textbf{variablen Kosten} (Wareneinkauf). Die Bezahlung der Mitarbeitenden über Tarif wird aktiv in den Gremien diskutiert, was auf ein Bewusstsein für faire Arbeitsbedingungen hindeutet.

\paragraph{Vertrauen und Transaktionskosten}
Die Genossenschaft reduziert Transaktionskosten durch ein hohes Maß an \textbf{Systemvertrauen}. Die transparenten und niedrigschwelligen Kommunikationsstrukturen (RocketChat, Plenum) und der Möglichkeit, verschiedene Newsletter zu abonnieren, ermöglichen es jedem Mitglied, mit wenig Aufwand informiert zu bleiben und Bedenken direkt zu äußern und an Entscheidungen mitzuwirken. Dies schafft Vertrauen und minimiert den Bedarf an aufwändigen Kontrollmechanismen.
