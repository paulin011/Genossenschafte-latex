\section{Einleitung}

Die Trink-Genosse eG stellt eine bemerkenswerte Fallstudie einer modernen Genossenschaft dar, die den traditionellen genossenschaftlichen Gedanken in ein zeitgemäßes städtisches Umfeld überträgt. Als eingetragene Genossenschaft betreibt sie die Bar \enquote{TRINK-Genossin} im Kölner Stadtteil Ehrenfeld und wird als \enquote{Kölns erste genossenschaftliche Bar} bezeichnet. Das primäre Ziel der Genossenschaft geht über den reinen Gastronomiebetrieb hinaus; sie versteht sich als ein Ort für vielfältige soziale und kulturelle Aktivitäten und verfolgt das explizite Ziel der Demokratieförderung. Die Trink-Genosse eG betont ihre \enquote{gelebte soziale Verantwortung} und fungiert als lebendiger Treffpunkt, an dem \enquote{Nachbarn, Freunde und Neugierige} zusammenkommen, um sich auszutauschen, zu diskutieren und gemeinsam zu feiern.

In einem Kontext, in dem traditionelle Wirtschaftsformen zunehmend hinterfragt werden und alternative Unternehmensmodelle an Bedeutung gewinnen, bieten Genossenschaften eine interessante Perspektive auf demokratische und nachhaltige Wirtschaftsgestaltung. Genossenschaften verkörpern die Verbindung von wirtschaftlicher Effizienz und sozialer Verantwortung und zeigen, wie demokratische Partizipation und gemeinschaftliches Engagement direkte wirtschaftliche und soziale Wertschöpfung ermöglichen können.

Die vorliegende Analyse verfolgt das Ziel, das Geschäftsmodell der Trink-Genosse eG systematisch zu strukturieren und zu typologisieren. Hierfür wird der in \textcite{blome-dreesGenossenschaftlicheGeschaeftsmodelleSemantik2023} vorgestellte morphologisch-typologische Ansatz und der zugehörige morphologische Kasten als theoretischer Rahmen herangezogen. Der Fokus liegt auf der Identifizierung spezifischer Ausprägungen der Genossenschaft in den vier Dimensionen des Geschäftsmodells: Mitglieder/Nutzer (\enquote{Wer?}), Nutzenversprechen (\enquote{Was?}), Wertschöpfungsarchitektur (\enquote{Wie?}) und Ertragsmechanik (\enquote{Wert?}).

Die zentrale Forschungsfrage lautet: \textit{Wie lässt sich das Geschäftsmodell der Trink-Genosse eG anhand des morphologischen Kastens nach Blome-Drees et al. (2023) charakterisieren und typologisch einordnen?}

Die Arbeit gliedert sich wie folgt: Nach der Einleitung wird in Kapitel 2 der theoretische Rahmen genossenschaftlicher Geschäftsmodelle dargestellt, einschließlich der morphologisch-typologischen Methode. Kapitel 3 erläutert das methodische Vorgehen der Fallstudienanalyse. Die empirischen Ergebnisse der Analyse des Geschäftsmodells der Trink-Genosse eG werden in Kapitel 4 präsentiert. Kapitel 5 diskutiert die typologische Einordnung und würdigt die Erkenntnisse kritisch. Das abschließende Kapitel 6 fasst die wichtigsten Ergebnisse zusammen und gibt einen Ausblick auf weiterführende Forschungsansätze.
