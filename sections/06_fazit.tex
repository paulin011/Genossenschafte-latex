\section{Fazit und Ausblick}

\subsection{Zusammenfassung der Ergebnisse}

Die vorliegende Analyse der Trink-Genosse eG mittels des morphologischen Kastens nach \textcite{blome-dreesGenossenschaftlicheGeschaeftsmodelleSemantik2023} hat gezeigt, dass moderne Genossenschaften deutlich komplexere und hybridere Geschäftsmodelle entwickeln können, als dies in traditionellen Typologien erfasst wird.

Die Trink-Genosse eG lässt sich als \textbf{produktionsorientiertes, bürgerschaftliches Geschäftsmodell mit starker Gemeinwohlorientierung} charakterisieren. Diese Charakterisierung verdeutlicht die Mehrdimensionalität ihres Ansatzes: Sie kombiniert die Produktion von Dienstleistungen mit bürgerschaftlichem Engagement und einer expliziten Gemeinwohlorientierung, die über die traditionelle Mitgliederförderung hinausgeht.

\subsection{Beantwortung der Forschungsfrage}

Die zentrale Forschungsfrage nach der Charakterisierung und typologischen Einordnung des Geschäftsmodells kann wie folgt beantwortet werden:

Das Geschäftsmodell der Trink-Genosse eG weist in allen vier Dimensionen des morphologischen Kastens hybride Ausprägungen auf:

\begin{itemize}
\item \textbf{Mitglieder/Nutzer (\enquote{Wer?}):} Kombination von Mitglieder- und Nichtmitgliederorientierung mit starker lokaler Verankerung
\item \textbf{Nutzenversprechen (\enquote{Was?}):} Integration wirtschaftlicher und sozio-kultureller Leistungen mit Mehrzweckcharakter
\item \textbf{Wertschöpfungsarchitektur (\enquote{Wie?}):} Multikanalstrategie mit starker Gewichtung immaterieller Ressourcen
\item \textbf{Ertragsmechanik (\enquote{Wert?}):} Diversifiziertes Erlösmodell mit Fokus auf ideelle Förderung statt Gewinnausschüttung
\end{itemize}

Typologisch handelt es sich um eine innovative Form der Bürgergenossenschaft, die traditionelle Grenzen zwischen Förder- und Produktivgenossenschaft sowie zwischen Mitglieder- und Gemeinwohlorientierung überschreitet.

\subsection{Implikationen für die Genossenschaftsforschung}

Die Ergebnisse haben mehrere Implikationen für die weitere Genossenschaftsforschung:

\begin{enumerate}
\item \textbf{Erweiterung des morphologischen Kastens:} Die Analyse zeigt Bereiche auf, in denen der bestehende morphologische Kasten um neue Dimensionen erweitert werden könnte, insbesondere im Bereich der Gemeinwohlorientierung und der digitalen Partizipation.

\item \textbf{Rolle immaterieller Ressourcen:} Die zentrale Bedeutung von Sozial- und Humankapital unterstreicht die Notwendigkeit, diese Dimensionen in künftigen Analysen stärker zu gewichten.
\end{enumerate}

\subsection{Ausblick und weiterführende Forschungsansätze}

Für die weitere Forschung ergeben sich mehrere interessante Ansatzpunkte:

\paragraph{Vergleichende Studien}
Eine vergleichende Analyse verschiedener urbaner Bürgergenossenschaften könnte Aufschluss über die Verallgemeinerbarkeit der hier identifizierten Muster geben.

\paragraph{Longitudinale Betrachtung}
Eine Langzeitstudie der Entwicklung der Trink-Genosse eG könnte wichtige Erkenntnisse über die Nachhaltigkeit und Evolution hybrider Genossenschaftsmodelle liefern.

\paragraph{Digitale Partizipation}
Die innovative Nutzung digitaler Tools für demokratische Prozesse verdient eine vertiefende Untersuchung, insbesondere im Hinblick auf ihre Auswirkungen auf Partizipation und Entscheidungsqualität.

\paragraph{Finanzielle Nachhaltigkeit}
Die Herausforderungen der finanziellen Nachhaltigkeit gemeinwohlorientierter Genossenschaften sollten systematisch untersucht werden, um Handlungsempfehlungen für die Praxis zu entwickeln.

\subsection{Schlussbemerkung}

Die Trink-Genosse eG demonstriert eindrucksvoll, wie genossenschaftliche Prinzipien in einem zeitgemäßen urbanen Kontext gelebt werden können. Sie zeigt auf, dass Genossenschaften nicht nur als wirtschaftliche Organisationen, sondern auch als Träger sozialen Wandels und demokratischer Innovation fungieren können. Ihr Erfolg basiert wesentlich auf der Fähigkeit, das Spannungsfeld zwischen wirtschaftlicher Notwendigkeit und ideellen Zielen produktiv zu gestalten und dabei neue Formen der Partizipation und Wertschöpfung zu entwickeln.

Für die Zukunft des genossenschaftlichen Sektors könnte das Modell der Trink-Genosse eG wegweisend sein: Es verbindet traditionelle genossenschaftliche Werte mit innovativen Ansätzen der Demokratieförderung und Gemeinwesenarbeit und zeigt damit neue Möglichkeiten für die gesellschaftliche Relevanz von Genossenschaften auf.
