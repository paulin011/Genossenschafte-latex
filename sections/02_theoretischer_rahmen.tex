\section{Theoretischer Rahmen: Genossenschaftliche Geschäftsmodelle}

\subsection{Semantik: Das Wesen der Genossenschaft}

Genossenschaften sind im Kern freiwillige Zusammenschlüsse von Personen, deren Mitglieder durch die Leistungen eines gemeinsam gegründeten und betriebenen Unternehmens in ihren wirtschaftlichen und sozialen Interessen gefördert werden sollen \autocite{blome-dreesGenossenschaftlicheGeschaeftsmodelleSemantik2023}. Der \enquote{Förderzweck} der Mitglieder bildet dabei die oberste Leitmaxime genossenschaftlichen Handelns. Der \enquote{Markterfolg} des genossenschaftlichen Geschäftsbetriebs ist hierbei nicht Selbstzweck, sondern eine \enquote{unbedingte Nebenbedingung} zur langfristigen Erfüllung dieses Förderzwecks \autocite{blome-dreesGenossenschaftlicheGeschaeftsmodelleSemantik2023}.

Ein konstitutives Merkmal von Genossenschaften ist das sogenannte \enquote{Identitätsprinzip}. Es beschreibt die doppelte Rolle der Mitglieder, die gleichzeitig Kapitalgeber und entweder Kunden, Lieferanten oder Beschäftigte des genossenschaftlichen Unternehmens sind. Dieses Prinzip gilt als entscheidendes Abgrenzungskriterium zu anderen Unternehmenstypen \autocite{blome-dreesGenossenschaftlicheGeschaeftsmodelleSemantik2023}. 

Darüber hinaus zeichnen sich Genossenschaften durch ihr \enquote{Demokratieprinzip} aus. Sie sind demokratisch verfasste Unternehmen, in denen das \enquote{Eine Person – Eine Stimme Prinzip} verwirklicht ist, unabhängig von der Höhe der Kapitalbeteiligung eines Mitglieds. Dies stellt ein wesentliches Unterscheidungsmerkmal zu erwerbswirtschaftlichen Unternehmen dar, bei denen die Stimmrechte oft an die Kapitalanteile gekoppelt sind \autocite{blome-dreesGenossenschaftlicheGeschaeftsmodelleSemantik2023}. 

Die Genossenschaften weisen zudem eine \enquote{Doppelnatur} auf, da sie sowohl als Wirtschaftsunternehmen mit marktbezogenen Funktionen als auch als Personenvereinigungen mit gruppenbezogenen Funktionen agieren und diese beiden Bereiche ausbalancieren müssen \autocite{blome-dreesGenossenschaftlicheGeschaeftsmodelleSemantik2023}.

\subsection{Morphologie genossenschaftlicher Geschäftsmodelle}

Die morphologisch-typologische Methode dient als wissenschaftliche Erkenntnismethode, um Merkmale realer Betriebe systematisch zu erfassen, deren spezifischen Sinn zu bestimmen und darauf aufbauend Typen anhand von Struktur- und Sinnunterschieden zu bilden \autocite{blome-dreesGenossenschaftlicheGeschaeftsmodelleSemantik2023}. Im Rahmen dieser Methode ist der \enquote{morphologische Kasten} eine weit verbreitete Darstellungsform. Er fungiert als tabellarische Matrixanordnung, die verschiedene erfasste Merkmale – also abgrenzbare Eigenschaften eines zu untersuchenden Objektbereichs – und deren jeweilige Ausprägungen, die ein Merkmal inhaltlich auskleiden, systematisiert \autocite{blome-dreesGenossenschaftlicheGeschaeftsmodelleSemantik2023}.

Die Bildung von Typen erfolgt durch die Kombination von Merkmalsausprägungen innerhalb des morphologischen Kastens. Hierbei wird ein \enquote{Leitmerkmal} verwendet, um die potenzielle Vielfalt an Ausprägungsmöglichkeiten in überschaubare Teilmengen zu untergliedern. Ein solches Leitmerkmal muss zwei wichtige Kriterien erfüllen: Es muss eine ausreichende Trennschärfe aufweisen, um eine eindeutige Zuordnung der Objekte zu gewährleisten, und es muss im Hinblick auf den Untersuchungszusammenhang von Relevanz sein, da es die Generalisierungsachsen und somit die Art der zulässigen Aussagen über die gebildeten Typen bestimmt \autocite{blome-dreesGenossenschaftlicheGeschaeftsmodelleSemantik2023}.

Im Kontext der vorliegenden Untersuchung wird das Konzept des \enquote{generativen Sprachspiels} für genossenschaftliche Geschäftsmodelle, wie es in \textcite{blome-dreesGenossenschaftlicheGeschaeftsmodelleSemantik2023} eingeführt wird, als übergeordneter Rahmen verstanden. Dieses Sprachspiel zielt darauf ab, grundlegende Denk- und Sprachkategorien zur Beschreibung des Objektbereichs bereitzustellen und die Generierung neuer Theorieentwürfe für die Führung von Genossenschaften zu ermöglichen.

\begin{sidewaystable}[htbp]
\centering
\small
\setlength{\tabcolsep}{4pt}
\renewcommand{\arraystretch}{1.3}
\caption{Morphologischer Kasten genossenschaftlicher Geschäftsmodelle. Eigene Darstellung.}
\label{tab:morphologischer_kasten}
\begin{tabularx}{\textwidth}{|p{0.18\textwidth}|p{0.20\textwidth}|p{0.20\textwidth}|X|p{0.08\textwidth}|}
\hline
\textbf{Merkmalsgruppe} & \textbf{Ordnungsmerkmal} & \textbf{Einzelmerkmal} & \textbf{Ausprägungen} & \textbf{Mischformen} \\
\hline
\multirow{14}{*}{Sinnbezogene Merkmale} & \multirow{8}{*}{Mitglieder/Nutzer (\enquote{Wer?})} & Leistungsadressaten & Mitglieder; Dritte; Gesellschaft & Ja \\
\cline{3-5}
 &  & Identit\"atsprinzip & Ja; Nein\\[-2pt]
 &  &  & Eigent\"umer \& Nutzer; Eigent\"umer \& Besch\"aftigte\\[-2pt]
 &  &  & F\"ordergenossenschaft; Produktivgenossenschaft & Nein \\
\cline{3-5}
 &  & Gesch\"aftsbeziehung & Hauptzweck; Nebenzweck\\[-2pt]
 &  &  & Mitgliedergesch\"aft; Nichtmitgliedergesch\"aft & Ja \\
\cline{3-5}
 &  & Tr\"agerschaft & Privat; Staatlich & Ja \\
\cline{3-5}
 &  & Betriebsformen & Haushalte; Unternehmen &  \\
\cline{2-5}
 & \multirow{6}{*}{Nutzenversprechen (\enquote{Was?})} & R\"aumliche Verankerung & Lokal; Regional; \"Uberregional; National; International & Ja \\
\cline{3-5}
 &  & Leistungsarten & Wirtschaftlich; Sozial\\[-2pt]
 &  &  & G\"uter; Dienstleistungen\\[-2pt]
 &  &  & Produktion; Bezug; Absatz & Ja \\
\cline{3-5}
 &  & Schl\"usselaktivit\"aten & \"Okonomisierung; Vertretung; Koordinierung & Ja \\
\cline{3-5}
 &  & Funktions\"ubernahme & Eine Funktion; Mehrere Funktionen & Ja \\
\hline
\multirow{10}{*}{Strukturbezogene Merkmale} & \multirow{6}{*}{Wertsch\"opfungsarchitektur} & Kooperationspartner & Verbundinterne Kooperationspartner; Verbundexterne Kooperationspartner\\[-2pt]
 &  &  & Finanzielle Beteiligung; Nicht-finanzielle Beteiligung & Ja \\
\cline{3-5}
 &  & Vertriebskan\"ale & Eigene Vertriebskan\"ale; Fremde Vertriebskan\"ale\\[-2pt]
 &  &  & Analog; Digital\\[-2pt]
 &  &  & Einkanalstrategie; Multikanalstrategie; Omnikanalstrategie\\[-2pt]
 &  &  & Filialen; Vertriebsabteilungen; Online-Shop; Plattformen & Ja \\
\cline{2-5}
 & \multirow{4}{*}{Ertragsmechanik (\enquote{Wert})} & Ressourcen & Materiell; Immateriell\\[-2pt]
 &  &  & Sachkapital; Finanzkapital; Sozialkapital; Humankapital & Ja \\
\cline{3-5}
 &  & Erl\"osmodell & Umsatzerl\"ose; Beteiligungserl\"ose; Regelm\"a\ss{}ige Beitr\"age; Subventionen & Ja \\
\cline{3-5}
 &  & Kostenmodell & Fixkosten; Variable Kosten & Ja \\
\hline
\end{tabularx}
\end{sidewaystable}


\subsection{Typologie von Genossenschaften}

Das morphologische System ermöglicht die Bildung verschiedener Genossenschaftstypen durch die systematische Kombination von Merkmalsausprägungen. Als Leitmerkmale dienen dabei die Schlüsselaktivitäten, die sich in drei Hauptkategorien unterteilen lassen:

\begin{itemize}
\item \textbf{Ökonomisierung:} Fokus auf wirtschaftliche Wertschöpfung und Effizienz
\item \textbf{Vertretung:} Interessenvertretung und Advocacy-Funktionen
\item \textbf{Koordinierung:} Organisation und Steuerung von Prozessen und Beziehungen
\end{itemize}

Diese Typologie wird durch weitere Merkmale wie die räumliche Verankerung (lokal, regional, überregional), die Trägerschaft (privat, staatlich) und die Funktionsübernahme (eine Funktion vs. mehrere Funktionen) differenziert. Besondere Bedeutung kommt dabei Bürgergenossenschaften zu, die sich durch ihre lokale Verankerung und die Bereitstellung von Infrastruktur für das Gemeinwesen auszeichnen.
