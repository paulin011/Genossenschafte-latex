\section{Methodik}

\subsection{Forschungsdesign}

Die vorliegende Untersuchung folgt einem qualitativen Forschungsansatz in Form einer Einzelfallstudie. Diese Methode eignet sich besonders gut für die tiefgreifende Analyse komplexer sozialer Phänomene in ihrem realen Kontext. Die Fallstudie der Trink-Genosse eG ermöglicht es, die theoretischen Konstrukte des morphologischen Kastens an einem konkreten Beispiel zu erproben und zu verfeinern.

Die Auswahl der Trink-Genosse eG als Untersuchungsgegenstand erfolgte aufgrund mehrerer Faktoren: Sie repräsentiert eine innovative Form der Genossenschaft im urbanen Kontext, weist eine starke sozio-kulturelle Orientierung auf und steht beispielhaft für moderne Formen des genossenschaftlichen Wirtschaftens, die über traditionelle Geschäftsmodelle hinausgehen.

\subsection{Datenerhebung}

Die Datenerhebung erfolgte mittels einer Triangulation verschiedener Quellen:

\begin{enumerate}
\item \textbf{Experteninterview:} Durchführung eines leitfadengestützten Interviews mit einem Aufsichtsratsmitglied der Trink-Genosse eG zur Vertiefung spezifischer Aspekte des Geschäftsmodells, der Organisationsstruktur und der strategischen Ausrichtung.

\item \textbf{Dokumentenanalyse:} Auswertung öffentlich zugänglicher Dokumente, einschließlich der Website der Genossenschaft, Presseberichte und weiterer Kommunikationsmaterialien.

\item \textbf{Teilnehmende Beobachtung:} Informelle Beobachtungen des Betriebs und der Aktivitäten der Bar zur Kontextualisierung der erhobenen Daten.
\end{enumerate}

Die Datenerhebung orientierte sich an den vier Dimensionen des morphologischen Kastens, um eine systematische Erfassung aller relevanten Merkmalsausprägungen zu gewährleisten.

\subsection{Datenauswertung}

Die Auswertung der erhobenen Daten erfolgte mittels qualitativer Inhaltsanalyse nach \textcite{mayring2010}. Dabei wurden die Daten entlang der Kategorien des morphologischen Kastens strukturiert und analysiert:

\begin{itemize}
\item \textbf{Mitglieder/Nutzer (\enquote{Wer?}):} Analyse der Leistungsadressaten, des Identitätsprinzips, der Geschäftsbeziehungen, Trägerschaft und räumlichen Verankerung
\item \textbf{Nutzenversprechen (\enquote{Was?}):} Untersuchung der Leistungsarten, Schlüsselaktivitäten und Funktionsübernahme
\item \textbf{Wertschöpfungsarchitektur (\enquote{Wie?}):} Betrachtung der Kooperationspartner und Vertriebskanäle
\item \textbf{Ertragsmechanik (\enquote{Wert?}):} Analyse der Ressourcen, Erlös- und Kostenmodelle
\end{itemize}

Die Zuordnung zu den jeweiligen Merkmalsausprägungen erfolgte durch systematischen Vergleich der empirischen Befunde mit den theoretischen Kategorien des morphologischen Kastens. Abweichungen und Besonderheiten wurden gesondert dokumentiert und in die typologische Einordnung einbezogen.
